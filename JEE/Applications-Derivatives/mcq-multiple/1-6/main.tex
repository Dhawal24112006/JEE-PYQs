\iffalse
 \title{16.Applications of Derivatives}
 \author{EE24btech11058-P.Shiny Diavajna}
 \section{mcq-multiple} 
\fi

    
%\begin{enumerate}
       \item 
	     Let $P\brak{x} = a_0+ a_1x^2+a_2x^4+\dots a_nx^{2n}$ be a polynomial in a real variable $x$ with \\
		 $0<a_0<a_1<a_2<\dots a_n. $ The function $P\brak{x}$ has 
	 \hfill(1986- 2 Marks)
      \begin{enumerate}

         \item neither a maximum nor a minimum
 
         \item only one maximum

         \item only one minimum

         \item only one maximum and only one minimum
   
         \item none of these\\

      \end{enumerate} 

      \item
       If the line $ax+by+c = 0$ is a normal to the curve $xy=1,$ then 
	 \hfill(1986-2 Marks)
        \begin{enumerate}
         \item $a>0,b>0$ 
         \item $a>0,b<0$ 
         \item $a<0,b>0$
         \item $a<0,b<0$
         \item none of these.
       \end{enumerate}

      \item 
      The smallest positive root of the equation, $\tan x-x=0$ lies in 
	\hfill(1987-2 Marks)
      \begin{enumerate}
       \item $\brak{0,\frac{\pi}{2}}$
       \item $\brak{\frac{\pi}{2},\pi}$
       \item $\brak{\pi,\frac{3\pi}{2}}$
       \item $\brak{\frac{3\pi}{2},2\pi}$
       \item None of these\\
      \end{enumerate}

        \item
	Let $f$ and $g$ be increasing and decreasing functions, respectively from $[0,\infty)$ to $[0,\infty)$. Let $h\brak{x} = f\brak{g\brak{x}}.$ If $h\brak{0} = 0,$ then $h\brak{x}-h\brak{1}$ is
	 \hfill(1987-2 Marks)
        \begin{enumerate}
          \item always zero
          \item always negative
          \item always positive
          \item strictly increasing
          \item None of these.\\
        \end{enumerate}

       \item 
       If 
	\begin{align*}
	 f\brak{x}=\begin{cases} 
	 3x^2+12x-1 & \text{if }-1 \le x\le 2\\
	 37-x & \text{if } 2<x \le 3 
         \end{cases}
       \end{align*}then:

	 \hfill(2008)
       \begin{enumerate}
	 \item $f\brak{x}$ is increasing on $\sbrak{-1,2}$
	 \item $f\brak{x}$ is continuous on $\sbrak{-1,3}$
	 \item $f^{\prime}\brak{2}$ does not exist
	 \item $f\brak{x}$ has the maximum value at $x=2$\\
       \end{enumerate}

     \item
	     Let $h\brak{x}$ = $f\brak{x}-(f\brak{x})^2+ (f\brak{x})^3$  for every real number $x.$ Then
     \hfill(1998-2 Marks)\\
     \begin{enumerate}
      \item $h$ is increasing whenever $f$ is increasing
      \item $h$ is increasing whenever $f$ is decreasing
      \item $h$ is decreasing whenever $f$ is decreasing
      \item nothing can be said in general.\\
     \end{enumerate}
 %\end{enumerate}

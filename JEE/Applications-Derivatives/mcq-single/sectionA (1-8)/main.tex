\iffalse
\title{Assignment}
\author{ee24btech11056-S.Kavya Anvitha}
\section{mcq-single} 
\fi

% \begin{enumerate}

\item If $a+b+c = 0$, then the quadratic equation $3ax^2 + 2bx + c = 0$
has \hfill$\brak{1983 - 1 Mark}$
\begin{enumerate}
	\item at least one root in\sbrak{0,1}
	\item one root in \sbrak{2,3} and other in \sbrak{-2,-1}
        \item imaginary roots
	\item none of these
\end{enumerate}

 \item $AB$ is a diameter of a circle and $\vec C$ is any point on the
circumference of the circle. Then
\hfill$\brak{1983 - 1 Mark}$
\begin{enumerate}
	\item the area of$\Delta ABC$ is maximum when it is isosceles
	\item the area of $\Delta ABC$ is minimum when it is isosceles
	\item the perimeter of $\Delta ABC$ is minimum when it is isosceles
	\item none of these
\end{enumerate}

\item The normal to the curve 
$x = a\brak{\cos \theta + \theta\sin \theta}$
$y = a\brak{\sin \theta - \theta\cos \theta}$
at any point '$\theta$' is such that \hfill$\brak{1983 - 1 Mark}$
\begin{enumerate}
	\item it makes  constant angle with the $x$ - axis
	\item it passes through the origin
	\item it is at a constant distance from the origin
	\item none of these
\end{enumerate}

\item If $y=a\ln x + bx^2 +x$ has its extremum values at 
$\vec x = -1$ and $\vec x = 2$, then


\hfill$\brak{1983 - 1 Mark}$

\begin{enumerate}
\begin{multicols}{2}
	\item $ a = 2$, $b = -1$
	\item $a = 2$, b = $\frac{-1}{2}$
	\item $a = -2$, b = $\frac{1}{2}$
	\item none of these
\end{multicols}
\end{enumerate}

\item Which one of the following curves cut the parabola
$y^2 = 4ax$ at right angles?
\hfill$\brak{1994}$
\begin{enumerate}
\begin{multicols}{2}
	\item $x^2 + y^2 = a^2$
        \item $e^{\frac{-x}{2a}}$
	\item $y = ax$
	\item $x^2 = 4ay$
\end{multicols}
\end{enumerate}

\item The function defined by 
$f\brak{x} = \brak{x+2}e^{-x}$ is
\hfill$\brak{1994}$
\begin{enumerate}
	\item decreasing for all $x$
	\item decreasing in $\brak{-\infty, -1}$ and increasing
		in $\brak{(-1, \infty)}$
        \item increasing for all $x$
	\item decreasing in $\brak{(-1, \infty)}$ and increasing
		in $\brak{(-\infty, -1)}$
\end{enumerate}

\item The function 
\begin{align*}
		f\brak{x} =\frac{\ln \brak{\pi + x}}{\ln \brak{e + x}}
\end{align*} is
\hfill$\brak{1995S}$

\begin{enumerate}
	\item increasing on $\brak{0, \infty}$
	\item decreasing on $\brak{0, \infty}$
	\item increasing on $\brak{0, \frac{\pi}{e}}$,
		decreasing on $\brak{\frac{\pi}{e}, \infty}$
	\item decreasing on $\brak{0, \frac{\pi}{e}}$,
		increasing on $\brak{\frac{\pi}{e}, \infty}$
\end{enumerate}

\item On the interval \sbrak{0, 1} the function $x^{25}\brak{1-x}^{25}$
takes its maximum value at the point 

\hfill$\brak{1995S}$
\end{enumerate}

\begin{enumerate}
\begin{multicols}{4}
	\item $0$ 
	\item $\frac{1}{4}$ 
	\item $\frac{1}{2}$ 
        \item $\frac{1}{3}$
\end{multicols}
\end{enumerate}

% \end{enumerate}



\iffalse
  \title{Assignment 1}
  \author{Srihaas Gunda}
  \section{integer}
\fi



\item The centres of two circles $C_1$ and $C_2$ each of unit radius are at a distance of $6$ units from each other. Let $P$ be the midpoint of the line segment joining the centres of $C_1$ and $C_2$ and $C$ be a circle touching circles $C_1$ and $C_2$ externally.If a common tangent to $C_1$ and $C$ passing through P is also a common tangent to $C_2$ and $C$, then the radius of circle $C$ is \hfill(2009)\\
\item The straight line $2x-3y=1$ divides the circular region $x^2+y^2\leq6$ into two parts.\\
If  S  is \{ $\brak{2,3/4},\brak{5/2,3/4},\brak{1/4,-1/4},\brak{1/8,1/4}$ \}  then the  number of point(s) in S lying inside the smaller part is \hfill(2011)\\
\item For how many values of $p$, the circle $x^2+y^2+2x+4y-p=0$ and the coordinate axes have exactly three common points? \hfill(JEE Adv. 2017)\\
\\
\item Let the point B be the reflection of the point A$\brak{2,3}$ with respect to the line $8x-6y-23=0$.Let $T_A$ and $T_B$ be circles of radii $2$ and $1$ with centres A and B respectively.Let T be a common tangent to the circles $T_A$ and $T_B$ such that both the circles are on the same side of T.If C is the point of intersection of T and the line passing through A and B,then the length of the line segment AC is \hfill(JEE Adv. 2019)



\iffalse
  \title{Assignment}
  \author{Jashwanth.Medamoni}
  \section{mcq-multiple}
\fi

%   \begin{enumerate}
    \item The number of the values of $c$ such that the straight line $y=4x+c$ touches the curves $(x^2/4)+y^2=1$ is \hfill(1998 - 2 Marks)\\
	\begin{enumerate}
			\begin{multicols}{2}
	\item $0$
	\item $1$
		\columnbreak
	\item $2$
	\item infinite
			\end{multicols}
	\end{enumerate}

\item If $\vec{P}=\brak{x,y}$, $\vec{F}_{1}=\brak{3,0}$, $\vec{F}_{2}=\brak{-3,0}$ and $16x^2+25y^2=400$, then $\vec{PF}_{1}+\vec{PF}_{2}$ equals \hfill(1998-2 Marks)\\
	\begin{enumerate}
			\begin{multicols}{2}
		\item $8$
		\item $6$
			\columnbreak
		\item $10$
		\item $12$
			\end{multicols}
	\end{enumerate}

\item On the ellipse $4x^2+9y^2=1$, the points at which the tangents are parallel to the line $8x=9y$ are \hfill(1999-3 Marks)\\
	\begin{enumerate}
			\begin{multicols}{2}
		\item $\brak{\frac{2}{5},\frac{1}{5}}$
		\item $\brak{-\frac{2}{5},\frac{1}{5}}$
			\columnbreak
		\item $\brak{-\frac{2}{5},-\frac{1}{5}}$
		\item $\brak{\frac{2}{5},-\frac{1}{5}}$
			\end{multicols}
	\end{enumerate}

\item The equations of the common tangents to the parabola $y=x^2$ and $y=-\brak{x-2}^2$ is/are \hfill(2006-5M,-1)\\
	\begin{enumerate}
			\begin{multicols}{2}
		\item $y=4\brak{x-1}$
		\item $y=0$
			\columnbreak
		\item $y=-4\brak{x-1}$
		\item $y=-30x-50$
			\end{multicols}
	\end{enumerate}

\item Let a hyperbola passes through the focus of the ellipse $\frac{x^2}{25}+\frac{y^2}{16}=1$. The transverse and conjugate axes of this hyperbola coincide with the major and minor axes of the given ellipse, also the product of eccentricities of given ellipse and hyperbola is $1$, then \hfill (2006-5M,-1)\\
	\begin{enumerate}
		\item the equation of hyperbola is $\frac{x^2}{9}-\frac{y^2}{16}=1$
		\item the equation of hyperbola is $\frac{x^2}{9}-\frac{y^2}{25}=1$
		\item focus of hyperbola is $\brak{5,0}$
		\item vertex of hyperbola is $\brak{5\sqrt{3},0}$
	\end{enumerate}

\item Let $\vec{P}\brak{x_{1}, y_{1}}$ and $\vec{Q}\brak{x_{2},y_{2}}$, $y_{1}<0,y_{2}<0$, be the end points of the latus rectum of the ellipse $x^2+4y^2=4$. The equations of parabolas with latus rectum $\vec{PQ}$ are \hfill(2008)\\
	\begin{enumerate}
			\begin{multicols}{2}
		\item $x^2+2\sqrt{3}y=3+\sqrt{3}$
		\item $x^2-2\sqrt{3}y=3+\sqrt{3}$
			\columnbreak
		\item $x^2+2\sqrt{3}y=3-\sqrt{3}$
		\item $x^2-2\sqrt{3}y=3-\sqrt{3}$
			\end{multicols}
	\end{enumerate}

\item In a traingle $\vec{ABC}$ with fixed base $\vec{BC}$, the vertex $\vec{A}$ moves such that
	$$\cos{B}+\cos{C}=4\sin^2{\frac{A}{2}}$$.
		If $a,b$ and $c$ denote the lengths of the sides of the traingle opposite to the angles $A,B$ and $C$, respectively, then \hfill(2009)\\
		\begin{enumerate}
			\item $b+c=4a$
			\item $b+c=2a$
			\item locus of the point $\vec{A}$ is an ellipse
			\item locus of the point $\vec{A}$ is a pair of straight lines
		\end{enumerate}

	\item The tangent $\vec{PT}$ and the normal $\vec{PN}$ to the parabola $y^2=4ax$ at a point $\vec{T}$ and $\vec{N}$, respectively. The locus of the centroid of the traingle $\vec{PTN}$ is a parabola whose \hfill(2009)\\
		\begin{enumerate}
				\begin{multicols}{2}
			\item vertex is $\brak{\frac{2a}{3},0}$
			\item directrix is $x=0$
				\columnbreak
			\item latus rectum is $\frac{2a}{3}$
			\item focus is $\brak{a,0}$
				\end{multicols}
		\end{enumerate}

\item An ellipse intersects the hyperbola $2x^2-2y^2=1$ orthogonally. The eccentricity of the ellipse is reciprocal of that of the hyperbola. If the axes of the ellipse are along the coordinate axes, then \hfill(2009)\\
		\begin{enumerate}
			\item equation of the ellipse is $x^2+2y^2=2$
			\item the foci of ellipse are $\brak{\pm1,0}$
			\item equation of the ellipse is $x^2+2y^2=4$
			\item the foci of ellipse are $\brak{\pm\sqrt{2},0}$
		\end{enumerate}

% \end{enumerate}

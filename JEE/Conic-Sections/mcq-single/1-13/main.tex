\iffalse
 \title{ASSIGNMENT-1}
 \author{EE24BTECH11029- J SHRETHAN REDDY}
 \section{mcq-single}
\fi

%\begin{enumerate}
    \item The equation $\frac{x^2}{1-r}-\frac{y^2}{1+r}=1,r\textgreater1$
represents:
          \hfill  \brak{1981-2 Marks}
\begin{enumerate}
    \item an ellipse    \item      b)   a hyperbola
   \item a circle     \item   d) none of there 
\end{enumerate}
\item Each of the four inequalities give below defines a region in $xy$ plane.One of these four regions does not have the following property.For any two points   $\brak{\frac{x_1+x_2}{2},\frac{y_1+y_2}{2}}$   is also in region.The inequality defining this region is:
         \hfill \brak{1981-2 Marks}
\begin{enumerate}
    \item $x^2+2y^2\le1$
    \item Max $\abs{x},\abs{y}$ $\le1$
    \item $x^2-y^2\le1$
    \item $y^2-x\le0$
\end{enumerate}
\item The equation $2x^2+3y^2-8x-18y+35=k$ represents:
        \hfill \brak{1994}
\begin{enumerate}
    \item no locus if $k\textless0$
    \item an ellipse if $k\textless0$
    \item a point if $k=0$
    \item a hyperbola if $k\textgreater0$ 
\end{enumerate}
\item Let $E$ be the ellipse $\frac{x^2}{9}+\frac{y^2}{4}=1$ and$C$ be the circle $x^2+y^2=9$.Let $P$ and $Q$ be the points $\brak{1,2}$ and $\brak{2,1}$ respectively.Then: 
        \hfill \brak{1994}
\begin{enumerate}
    \item $Q$ lies inside $C$ but outside $E$
    \item $Q$ lies outside both $C$ and $E$
    \item $P$ lies inside both $C$ and $E$
    \item $p$ lies inside $C$ but outside $E$ 
\end{enumerate}
\item Consider a circle with its centre lying on focus of the parabola $y^2=2px$ such that it touches the directrix of the parabola. Then a point of intersection of the circle and the parabola is
        \hfill\brak{1995S}
\begin{enumerate}
    \item $\brak{\frac{p}{2},p}$ or $\brak{\frac{p}{2},-p}$
    \item $\brak{\frac{p}{2},-\frac{p}{2}}$
    \item $\brak{-\frac{p}{2},p}$
    \item $\brak{-\frac{p}{2},-\frac{p}{2}}$
\end{enumerate}
\item The radius of the circle passing through the foci of the ellipse $\frac{x^2}{16}+\frac{y^2}{9}=1$. and having its centre at $\brak{0,3}$ is:
       \hfill \brak{1995S}
\begin{enumerate}
    \item $4$
    \item $3$
    \item $\sqrt{\frac{1}{2}}$
    \item $\frac{7}{2}$
\end{enumerate}
\item Let $P\brak{a\sec\theta,b\tan\theta}$ and $Q\brak{a\sec\phi,b\tan \phi}$,where $\theta+\phi=\pi/2$, be two points on the hyperbola $\frac{x^2}{a^2}-\frac{y^2}{b^2}=1$.If $\brak{h,k}$ is the point 0f intersection of the normals at $P$ and $Q$, then $K$ equal to 
      \hfill \brak{1999-2 Marks}
\begin{enumerate}
    \item $\frac{a^2+b^2}{a}$
    \item $-\brak{\frac{a^2+b^2}{a}}$
    \item $\frac{a^2+b^2}{b}$
    \item $-\brak{\frac{a^2+b^2}{b}}$
\end{enumerate}
\item If $x=9$ is the chord of contact of the hyperbola $x^2-y^2=9$,then the equation of the corresponding pair of tangents is:
    \hfill \brak{1999-2 Marks}
\begin{enumerate}
    \item $9x^2-8y^2+18x-9=0$
    \item $9x^2-8y^2-18x+9=0$
    \item $9x^2-8y^2-18x-9=0$
    \item $9x^2-8y^2+18x+9=0$
\end{enumerate}
\item The curve described parametrically by $x=t^2+t+1$,$y=t^2-t+1$ represents
     \hfill\brak{1999-2 Marks}
\begin{enumerate}
    \item a pair of straight lines
    \item an ellipse
    \item a parabola
    \item a hyperbola
\end{enumerate}
\item If $x+y=k$ is normal $y^2=12x$,then $K$ is
     \hfill\brak{2000s}
\begin{enumerate}
    \item $3$
    \item $9$
    \item $-9$
    \item $-3$
\end{enumerate}
\item If the line $x-1=0$ is the directrix of parabola $y^2-kx+8=0$,than one of the values of $K$ is
      \hfill\brak{2000S}
\begin{enumerate}
    \item $\frac{1}{8}$
    \item $8$
    \item $4$
    \item $\frac{1}{4}$ 
\end{enumerate}
\item The equation of the common tangent touching the circle $\brak{x-3}^2-kx+8=0$ and the parabola $y^2=4x$ above the $x$-axis is 
      \hfill\brak{2000s}
\begin{enumerate}
    \item $\sqrt{3}y=3x+1$
    \item $\sqrt{3}y=-\brak{x+3}$
    \item $\sqrt{3}y=x+3$
    \item $\sqrt{3}y=-\brak{3x+1}$
\end{enumerate}
    \item The equation of the directrix of the parabola $y^2+4y+4x+2=0$ is 
     \hfill \brak{2001S}
\begin{enumerate}
    \item $x=-1$
    \item $x=1$
    \item $x=-\frac{3}{2}$
     \item $x=\frac{3}{2}$
\end{enumerate}
%\end{enumerate}

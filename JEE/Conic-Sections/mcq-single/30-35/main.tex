
\iffalse
  \title{Assignment-1}
  \author{Jashwanth.Medamoni}
  \section{mcq-single}
\fi

%   \begin{enumerate}
    \item The normal at a point $\vec{P}$ on the ellipse $x^2 +4y^2=16$ meets the $x$-axis at $\vec{Q}$. If $\vec{M}$ is the mid point of the line segment $\vec{PQ}$, then the locus of $\vec{M}$ interests the latusrectums of the given ellipse at the points
	
	\hfill (2009)
		\begin{enumerate}
				\begin{multicols}{2}
			\item $\brak{\pm\frac{3\sqrt{5}}{2},\pm\frac{2}{7}}$
			\item $\brak{\pm\frac{3\sqrt{5}}{2},\pm\sqrt{\frac{19}{4}}}$
				\columnbreak
			\item $\brak{\pm2\sqrt{3},\pm\frac{1}{7}}$
			\item $\brak{\pm2\sqrt{3},\pm\frac{4\sqrt{3}}{7}}$
				\end{multicols}
		\end{enumerate}
		
\item The locus of the orthocentre of the traingle formed by the lines
		
			$$\brak{1+p}x-py+p\brak{1+p}=0,$$
			$$\brak{1+q}x-qy+q\brak{1+q}=0,$$
		and $y=0$, where $p \neq q$, is
		\hfill(2009)
\begin{enumerate}
		\begin{multicols}{2}
	\item a hyperbola
	\item a parabola
		\columnbreak
	\item an ellipse
	\item a straight line
		\end{multicols}   
\end{enumerate}

\item Let $\vec{P}\brak{6,3}$ be a point on the hyperbola $\frac{x^2}{a^2}-\frac{y^2}{b^2}=1$. If the normal at the point $\vec{P}$ intersects the $x$-axis at $\brak{9,0}$, then the eccentricity of the hyperbola is 
	\hfill (2011)\\
		\begin{enumerate}
				\begin{multicols}{2}
			\item$\sqrt{\frac{5}{2}}$
			\item$\sqrt{\frac{3}{2}}$
				\columnbreak
			\item$\sqrt{2}$
			\item$\sqrt{3}$
				\end{multicols}
		\end{enumerate}

	\item Let $\brak{x,y}$ be any point on the parabola $y^2=4x$. Let $\vec{P}$ be the point that divides the line segment from $\brak{0,0}$ to $\brak{x,y}$ in the ratio $1:3$. Then the locus of $\vec{P}$ is  \hfill(2011)\\
		\begin{enumerate}
				\begin{multicols}{2}
			\item $x^2=y$
			\item $y^2=2x$
				\columnbreak
			\item $y^2=x$
			\item $x^2=2y$
				\end{multicols}
		\end{enumerate}

	\item The ellipse $E_{1}$:$\frac{x^2}{9}+\frac{y^2}{4}=1$ is inscribed in a rectangle $\vec{R}$ whose sides are parallel to the coordinate axes. Another ellipse $E_{2}$ passing through the point $\brak{0,4}$ circumscribes the rectangle $\vec{R}$.The eccentricity of the ellipse $E_{2}$ is \hfill(2012)\\

		\begin{enumerate}
				\begin{multicols}{2}
			\item $\frac{\sqrt{2}}{2}$
			\item $\frac{\sqrt{3}}{2}$
				\columnbreak
			\item $\frac{1}{2}$
			\item $\frac{3}{4}$
				\end{multicols}|
		\end{enumerate}

	\item The common tangents to the circie $x^2+y^2=2$ and the parabola $y^2=8x$ touch the circle at the points $\vec{P}$, $\vec{Q}$ and the parabola at the points $\vec{R}$, $\vec{S}$.Then the area of the quadrilateral $\vec{PQRS}$ is \hfill(JEE Adv. 2014)\\
		\begin{enumerate}
				\begin{multicols}{2}
			\item $3$
			\item $6$
				\columnbreak
			\item $9$
			\item $15$
				\end{multicols}
		\end{enumerate}

% \end{enumerate}

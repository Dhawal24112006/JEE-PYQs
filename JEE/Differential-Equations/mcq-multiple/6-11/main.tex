\iffalse
\title{ASSIGNMENT 2}
\author{Akshara Sarma Chennubhatla}
\section{mcq-multiple}
\fi
% \begin{enumerate}[start = 6]
\item Let $y\brak{x}$ be a solution of the differential equation $\brak{1+e^x}y^\prime + ye^x = 1$. If $y\brak{0} = 2$, then which of the following statement is (are) true?
\hfill{\brak{JEE Adv.2015}}
\begin{enumerate}
\begin{multicols}{2}
\item $y\brak{-4} = 0$
\columnbreak
\item $y\brak{-2} = 0$
\end{multicols}
\item $y\brak{x}$ has a critical point in the interval \\ $\brak{-1,0}$
\item $y\brak{x}$ has no critical point in the interval \\ $\brak{-1,0}$
\end{enumerate}
\item Consider the family of all circles whose centers lie on the straight line $y = x$. If this family of circle is represented by the differential equation $Py^{\prime\prime} + Qy^\prime + 1 = 0$, where P, Q are functions of $x,y$ and $y^\prime$
\begin{align}
\brak{here\;y^\prime = \frac{dy}{dx}, y^{\prime\prime} = \frac{d^2y}{dx^2}},
\end{align} then which of the following statements is (are) true?
\hfill{\brak{JEE Adv.2015}}
\begin{enumerate}
\begin{multicols}{2}
\item $P = y + x$
\columnbreak
\item $P = y-x$
\end{multicols}
\begin{multicols}{2}
\item $P+Q = 1-x+y\\+y^\prime+\brak{y^\prime}^2$
\columnbreak
\item $P-Q=x+y\\-y^\prime-\brak{y^\prime}^2$ 
\end{multicols}
\end{enumerate}
\item Let $f:\brak{0,\infty} \rightarrow \Re$ be a differentiable function such that 
\begin{align}
f^\prime\brak{x} = 2-\frac{f\brak{x}}{x}
\end{align} for all $x \in \brak{0,\infty} and f\brak{1} \neq 1$. \\Then
\hfill{\brak{JEE Adv.2016}}
\begin{enumerate}
\begin{multicols}{2}
\item $\lim\limits_{x \to 0+} f^\prime\brak{\frac{1}{x}} = 1$
\columnbreak
\item $\lim\limits_{x \to 0+} x f^\prime\brak{\frac{1}{x}} = 2$
\end{multicols}
\begin{multicols}{2}
\item $\lim\limits_{x \to 0+} x^2 f^\prime\brak{x} = 0$
\columnbreak
\item $\abs{f\brak{x}} \leq 2$ for all \\$x \in \brak{0,2}$ 
\end{multicols} 
\end{enumerate}
\item A solution curve of the differential equation 
\begin{align}
\brak{x^2+xy+4x+2y+4}\frac{dy}{dx}-y^2=0,\;x>0,
\end{align} passes through the point $\brak{1,3}$. Then the solution curve
\hfill{\brak{JEE Adv.2016}}
\begin{enumerate}
\item intersects $y=x+2$ exactly at one point
\item intersects $y=x+2$ exactly at two points
\item intersects $y=\brak{x+2}^2$ 
\item does NOT intersect $y=\brak{x+3}^2$
\end{enumerate}
\item Let $f:[0,\infty)\rightarrow \Re$ be a continuous \\function such that
\begin{align}
f\brak{x} = 1-2x+ \int_0^x e^{x-t} f(t) dt
\end{align} for all $x\in[0,\infty)$. Then, which of the following statement(s) is (are) TRUE?
\hfill{\brak{JEE Adv.2018}}
\begin{enumerate}
\item The curve $y=f\brak{x}$ passes through the \\point $\brak{1,2}$
\item The curve $y=f\brak{x}$ passes through the \\point $\brak{2,-1}$
\item The area of the region 
\begin{align}
\cbrak{\brak{x,y} \in \sbrak{0,1}\times\Re:f\brak{x}\leq y \leq \sqrt{1-x^2}} 
\end{align}
$$\text{is} \;\frac{\pi-2}{4}$$
\item The area of the region 
\begin{align}
\cbrak{\brak{x,y} \in \sbrak{0,1}\times\Re:f\brak{x}\leq y \leq \sqrt{1-x^2}}
\end{align} 
$$\text{is}\; \frac{\pi-1}{4}$$
\end{enumerate}
\item Let $\Gamma$ denote a curve $y=y\brak{x}$ which is in the first quadrant and let the point $\brak{1,0}$ lie on it. Let the tangent to $\Gamma$ at a point $P$ intersect the $y-axis$ at $Y_p$. If $PY_p$ has length 1 for each point $P$ on $\Gamma$, then which of the following options is/are correct?
\hfill{\brak{JEE Adv.2019}}
\begin{enumerate}
\item $y=-\log_e{\brak{\frac{1+\sqrt{1-x^2}}{x}}} + \sqrt{1-x^2}$
\item $xy^\prime-\sqrt{1-x^2} = 0$
\item $y=\log_e{\brak{\frac{1+\sqrt{1-x^2}}{x}}}-\sqrt{1-x^2}$
\item $xy^\prime+\sqrt{1-x^2} = 0$
\end{enumerate}
% \end{enumerate}

\iffalse
\title{22.MISCELLANEOUS}
\author{AI24BTECH11006 - Bugada Roopansha}
\section{mcq-single}
\fi
%\begin{enumerate}[start=16]
\item With two forces acting at point, the maximum affect is obtained when their resultant is $4$N $\cdot$If they act at right angles,then their resultant is $3$N$\cdot$Then the forces are
\hfill{[2004]}
\begin{enumerate}
\item $\brak{2+\frac{1}{2}\sqrt{3}}N and\brak{2-\frac{1}{2}\sqrt{3}}N$
\item $\brak{2+\sqrt{3}}N and \brak{2-\sqrt{3}}N$
\item  $\brak{2+\frac{1}{2}\sqrt{2}}N and\brak{2-\frac{1}{2}\sqrt{2}}N$
\item $\brak{2+\sqrt{2}}N and \brak{2-\sqrt{2}}N$
\end{enumerate}
\item In a right angle $\triangle$ABC , $\angle A$=90\degree and sides a,b,c are respectively $5cm,4cm and 3cm .$If a force $ \vec{F}$ has moments 0,9 and 16 in N cm units respectively about vertices A,B and C, then magnitude of $\vec{F}$ is
\hfill{[2004]}
\begin{enumerate}
\item $9$
\item $4$
\item $5$
\item $3$
\end{enumerate}
\item Three forces $\vec{P},\vec{Q} and\vec{R}$ acting along IA,IB and IC,where I is the incentre of $\triangle{ABC}$ are in equilibrium.Then $\vec{P} \colon \vec{Q} \colon \vec{R}$ is
\hfill{[2004]}
\begin{enumerate}
\item $\cosec{\frac{A}{2}}\colon\cosec{\frac{B}{2}}\colon\cosec{\frac{C}{2}}$
\item $\sin{\frac{A}{2}}\colon\sin{\frac{B}{2}}\colon\sin{\frac{C}{2}}$
\item $\sec{\frac{A}{2}}\colon\sec{\frac{B}{2}}\colon\sec{\frac{C}{2}}$
\item $\cos{\frac{A}{2}}\colon \cos{\frac{B}{2}}\colon \cos{\frac{C}{2}}$
\end{enumerate}
\item A particle moves towards east from point A to a point B at the rate of 4 kmph and then towards north from B to C at the rate of 5 kmph.If AB =12km and BC=5km , then its average speed for its journey from A to C are respectively
\hfill{[2004]}
\begin{enumerate}
\item $\frac{13}{9}kmph$ and $\frac{17}{9}kmph$
\item $\frac{13}{4}kmph$ and $\frac{17}{4}kmph$
\item $\frac{17}{9}kmph$ and $\frac{13}{9}kmph$
\item $\frac{17}{9}kmph$ and $\frac{13}{9}kmph$
\end{enumerate}
\item A velocity $\frac{1}{4}$m/s is resolved into two components along OA and OB making angles $30\degree and 45\degree$ respectively with the given velocity. Then the component along OB is
\hfill{[2004]}
\begin{enumerate}
\item $\frac{1}{8}\brak{\sqrt{6}-\sqrt{2}}m/s$
\item $\frac{1}{4}\brak{\sqrt{3}-1}m/s$
\item $\frac{1}{4} m/s$
\item $\frac{1}{8} m/s$
\end{enumerate}
\item If $t_1$ and $t_2$ are the times of flight of two particles having the same initial velocity u and range R on the horizontal. Then $ {t_1}^2 + {t_2}^2$ is equal to
\hfill{[2004]}
\begin{enumerate}
\item 1
\item $\frac{4{u}^2}{{g}^2}$
\item $\frac{{u}^2}{2g}$
\item $\frac{{u}^2}{g}$
\end{enumerate}
\item
Let R={\brak{3,3},\brak{6,6},\brak{9,9},\brak{12,12},\brak{6,12},\brak{3,9},\brak{3,12}\brak{3,6}} be a relation set A={3,6,9,12}.The relation is
\hfill{[2005]}
\begin{enumerate}
\item reflexive and transitive only
\item reflexive only
\item an equivalence relation
\item reflexive and symmetric only
\end{enumerate}
\item ABC is a triangle. Forces$\vec{P},\vec{Q},\vec{R}$ acting along IA,IB,IC respectively are in equilibrium, where I is the incentre of $\triangle ABC$. Then $P\colon Q \colon R$ is
\hfill{[2005]}
\begin{enumerate}
\item $\sin {A} \colon \sin {B} \colon \sin {C}$
\item $\sin{\frac{A}{2}}\colon \sin{\frac{B}{2}}\colon \sin{\frac{C}{2}}\colon$
\item $\cos{\frac{A}{2}}\colon \cos{\frac{B}{2}}\colon \cos{\frac{C}{2}}$
\item $\cos {A} \colon \cos{B}\colon \cos{C}$
\end{enumerate}
\item If in a frequency distribution, the mean and median are $21 and 22$ respectively,then its mode is approximately
\hfill{[2005]}
\begin{enumerate}
\item $22.0$
\item $20.5$
\item $25.5$
\item $24.0$
\end{enumerate}
\item A lizard ,at an initial distance of $21 cm$ behind an insect, moves from rest with an acceleration of $2cm/{s}^2$ and pursues the insect uniformly along a straight line at a speed of $20cm/s$. Then the lizard will catch the insect after
\hfill{[2005]}
\begin{enumerate}
\item $20s$
\item $1s$
\item $21s$
\item $24s$
\end{enumerate}
\item Two points A and B move from rest along a straight line with constant acceleration $f$ and $f'$ respectively. If A takes m sec more than B and describes 'n' units more than B in acquiring the same speed then 
\hfill{[2005]}
\begin{enumerate}
\item $\brak{f-f\prime}m^2 =ff\prime n$
\item $\brak{f+f\prime}m^2 =ff\prime n$
\item $\frac{1}{2}\brak{f+f\prime}m =ff\prime n^2$
\item $\brak{f\prime-f}n =ff\prime m^2$
\end{enumerate}
\item A and B are two like parallel forces.A couple of moment H lies in the plane of A and B and is contained with them.The resultant of A and B after combining is displaced through a distance
\hfill{[2005]}
\begin{enumerate}
\item $\frac{2H}{A-B}$
\item $\frac{H}{A+B}$
\item $\frac{H}{2\brak{A+B}}$
\item $\frac{H}{A-B}$
\end{enumerate}
\item Let $x_1,x_2 \dots x_n$ be n observations such that $\sum {x_i}^2 = 400$ and $\sum {x_i} = 80$. Then the possible value of n among the following is
\hfill{[2005]}
\begin{enumerate}
\item $15$
\item $18$
\item $9$
\item $12$
\end{enumerate}
\item A particle is projected from a point $O$ with na velocity u at an angle 60\degree with the horizontal.When it is moving in a direction at right angles to its direction at $O$, its velocity the is given by
\hfill{[2005]}
\begin{enumerate}
\item $\frac{u}{3}$
\item $\frac{u}{2}$
\item $\frac{2u}{3}$
\item $\frac{u}{\sqrt{3}}$
\end{enumerate}
\item The resultant R of two forces acting on a particle is at right angles to one of them and its magnitude is one third of the other force.Then the ratio of larger force to the smaller one is 
\hfill{[2005]}
\begin{enumerate}
\item $ 2\colon 1$
\item $ 3\colon \sqrt{2}$
\item $ 3\colon 2$
\item $ 3\colon 2\sqrt{2}$
\end{enumerate}



%\end{enumerate}


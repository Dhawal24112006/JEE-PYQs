\iffalse
    \title{Assignment}
    \author{Pendem nitesh sri satya - AI24BTECH11026}
    \section{mcq-single}
  \fi
%\begin{enumerate}

 \item Let $a$, $b$, $c$ be real numbers, a $\neq$ 0. If $\alpha$ is a root of $a^{2}x^{2}+bx+c=0$. $\beta$ is the root of $a^{2}x^{2}-bx-c=0$ and 0$<$$\alpha$$<$$\beta$, then the equation $a^{2}x^{2}+2bx+2c=0$ has a root $\gamma$ that always satisfies  \hfill \brak{1989 - 2Marks}
 \begin{multicols}{2}
\begin{enumerate}
\item $\gamma=\frac{\alpha+\beta}{2}$  
\item $\gamma=\alpha+\frac{\beta}{2}$
\item $\gamma=\alpha$ 
\item $\alpha<\gamma<\beta$
\end{enumerate}
\end{multicols}
 \item The number of solutions of the equation $\sin{(e)^{x}}=5^{x}+5^{-x}$ is \hfill  \brak{1990 - 2 Marks}
 \begin{multicols}{2}
 \begin{enumerate}
\item 0  
\item 1          
\item 2  
\item Infinitely many
\end{enumerate}
\end{multicols}
 \item Let $\alpha$,$\beta$ be the roots of the equation $(x-a)(x-b)=c$, $c \neq 0$ Then the roots of the equation $(x-a)(x-b)+c=0$ are \hfill  \brak{1992 - 2 Marks}
 \begin{multicols}{2}
 \begin{enumerate}
\item $a,c$ 
\item $b,c$
\item $a,b$ 
\item $a+c,b+c$
\end{enumerate}
\end{multicols}
 \item The number of point of intersection of two curves $y=2\sin{x}$ and $y=5x^{2}+2x+3$ is \hfill \brak{1994}
 \begin{multicols}{2}
 \begin{enumerate}
\item 0 
\item 1
\item 2  
\item $\infty$
\end{enumerate}
\end{multicols}
 \item If $p,q,r$ are +ve and are in A.P.,the roots of quadratic equation $px^{2}+qx+r=0$ are all real for \hfill \brak{1994}
 \begin{multicols}{2}
\begin{enumerate}
\item $\abs{\frac{r}{p}-7} \geq 4\sqrt{3}$  
\item $\abs{\frac{p}{r}-7} \geq 4\sqrt{3}$
\item all $p$ and $r$ 
\item no $p$ and $r$
\end{enumerate}
\end{multicols}
 \item Let $p,q \in {1, 2, 3, 4}$  The number of equations of the form $px^2+qx+1=0$ having real roots is \hfill  \brak{1994}
\begin{multicols}{2}
\begin{enumerate}
\item 15 
\item 9
\item 7 
\item 8
\end{enumerate}
\end{multicols}
 \item If the roots of the equation $x^{2}-2ax+a^{2}+a-3=0$ are real and less than 3, then \hfill \brak{1999 - 2 Marks}
\begin{multicols}{2}
\begin{enumerate}
\item $a<2$ 
\item $2 \leq a \leq 3$
\item $3 < a \leq 4$ 
\item $a > 4$
\end{enumerate}
\end{multicols}
 \item If $\alpha$ and $\beta$ ($\alpha < \beta$) are the roots of the equation $x^{2}+bx+c=0$, where $c < 0 < b$, then \hfill  \brak{2000S}
\begin{multicols}{2}
\begin{enumerate}
\item $0 < \alpha < \beta$ 
\item $\alpha < 0 < \beta < \abs{\alpha}$
\item $\alpha < \beta <0$ 
\item $\alpha < 0 < \abs{\alpha} < \beta$
\end{enumerate}
\end{multicols}
 \item If $a, b, c, d$ are positive real numbers such that $a+b+c+d=2$, then $M=(a+b)(c+d)$ satifies the relation \hfill \brak{2000S}
\begin{multicols}{2}
\begin{enumerate}
 \item $0 \leq M \leq 1$ 
 \item $1 \leq M \leq 2$
 \item $2 \leq M \leq 3$ 
 \item $3 \leq M \leq 4$
 \end{enumerate}
\end{multicols}
 \item If $b > a$, then the equation $(x-a)(x-b)-1=0$ has \hfill  \brak{2000S}
\begin{enumerate}
\item both roots in $(a,b)$
\item both roots in $(-\infty,a)$
\item both roots in $(b,+\infty)$
\item one root in $(-\infty,a)$ and the other in $(b,+\infty)$
\end{enumerate}
 \item  For the equation $3x^2+px+3=0$,$p>0$, if one of the root is square of the other, then $p$ is equal to \hfill \brak{2000S}
\begin{multicols}{2}
\begin{enumerate}
\item $\frac{1}{3}$ 
\item 1
\item 3 
\item $\frac{2}{3}$
 \end{enumerate}
\end{multicols}
 \item  If $a_1,a_2.....,a_n$ are positive real numbers whose product is a fixed number c, then the minimum value of $a_1+a_2+.......+a_{n-1}+2a_n$ \hfill \brak{2002S}
\begin{multicols}{2}
\begin{enumerate}
\item $n(2c)^{\frac{1}{n}}$ 
\item $(n+1)c^{\frac{1}{n}}$
\item $2nc^{\frac{1}{n}}$ 
\item $(n+1)(2n)^{\frac{1}{n}}$
 \end{enumerate}
 \end{multicols}
 \item The set of all real numbers $x$ for which $x^2-\abs{x+2}+x>0$, is \hfill \brak{2002S}
\begin{enumerate}
\item $(-\infty,-2) \cup (2,\infty)$ 
\item $(-\infty,-\sqrt{2}) \cup (\sqrt{2},\infty)$
\item $(-\infty,-1) \cup (1,\infty)$ 
\item $(\sqrt{2},\infty)$
\end{enumerate}
 \item  If $\alpha \in (0,\frac{\pi}{2})$ then $\sqrt{x^2+x}+\frac{\tan^{2}{\alpha}}{\sqrt{x^2+x}}$ is always greater than or equal to \hfill \brak{2003S}
\begin{multicols}{2}
\begin{enumerate}
\item 2$\tan{\alpha}$ 
\item 1
\item 2 
\item $\sec^{2}{\alpha}$
\end{enumerate}
\end{multicols}
 \item For all $'x'$ ,$x^{2}+2ax+10-3a>0$, then the interval in which $'a'$ lies is \hfill \brak{2004S}
\begin{multicols}{2}
\begin{enumerate}
\item $a<-5$ 
\item $-5<a<2$
\item $a>5$ 
\item $2<a<5$
\end{enumerate}
\end{multicols}

%\end{enumerate}



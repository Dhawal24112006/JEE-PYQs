\iffalse
  \title{Assignment}
  \author{AI24BRECH11025 PEDAPROLU LAKSHMI KUSHAL}
  \section{mcq-single}
\fi

%   \begin{enumerate}
    \item If one root is square of the other root of the equation $x^{2}+px+q=0$, then the relationship between p and q is \hfill (2004S)

\begin{enumerate}
    \item $p^{3}-q\brak{3p-1}+q^{2}=0$
    \item $p^{3}-q\brak{3p+1}+q^{2}=0$
    \item $p^{3}+q\brak{3p-1}+q^{2}=0$
    \item $p^{3}+q\brak{3p+1}+q^{2}=0$
\end{enumerate}
\item Let a,b,c be the sides of the triangle where $a\neq b\neq c$ and $\lambda  R$. If the roots of the equation 

$x^{2}+2\brak{a+b+c}x+3\lambda\brak{ab+bc+ca}=0$ are real, then \hfill(2006-3M,-1)
\begin{multicols}{2}
    \begin{enumerate}
        \item $\lambda<\frac{4}{3}$
        \item $\lambda>\frac{5}{3}$
        \item $\lambda \brak{\frac{1}{3},\frac{5}{3}}$
        \item $\lambda \brak{\frac{1}{3},\frac{4}{3}}$
    \end{enumerate}
\end{multicols}
\item Let $\alpha,\beta$ be the roots of the equation $x^{2}-px+r=0$ and $\frac{\alpha}{2},2\beta$ be the roots of the equation $x^{2}-qx+r=0$. Then the value of r is \hfill (2007-1marks)

\begin{enumerate}
    \item $\frac{2}{9}\brak{p-q}\brak{2q-p}$
    \item $\frac{2}{9}\brak{q-p}\brak{2p-q}$
    \item $\frac{2}{9}\brak{q-2p}\brak{2q-p}$
    \item $\frac{2}{9}\brak{2p-q}\brak{2q-p}$
\end{enumerate}
\item let p and q be real numbers such that $p\neq 0,p^{3}\neq q and p^{3}\neq -q.$ If $\alpha and \beta$ are nonzero complex numbers satisfying $\alpha+\beta=-p and \alpha^{3}+\beta{3}=q$,then a quadratic equation having $\frac{\alpha}{\beta} and \frac{\beta}{\alpha}$ as its roots 

is \hfill (2010)
\begin{enumerate}
    \item $\brak{p^{3}+q}x^{2}-\brak{p^{3}+2q}x+\brak{p^{3}+q}=0$
    \item $\brak{p^{3}+q}x^{2}-\brak{p^{3}-2q}x+\brak{p^{3}+q}=0$
    \item $\brak{p^{3}+q}x^{2}-\brak{5p^{3}-2q}x+\brak{p^{3}-q}=0$
    \item $\brak{p^{3}+q}x^{2}-\brak{5p^{3}+2q}x+\brak{p^{3}-q}=0$
\end{enumerate}
\item Let $\brak{x_0,y_0}$ be the solution of the following equatoions

$\brak{2x}^{ln2}=\brak{3y}^{ln3}$

$3^{lnx}=2^{lny}$

then $x_{0}$ is \hfill (2011)
\begin{multicols}{4}
\begin{enumerate}
    \item $\frac{1}{6}$
    \item $\frac{1}{3}$
    \item $\frac{1}{2}$
    \item 6
\end{enumerate}
    
\end{multicols}
\item Let $\alpha and \beta$ be the roots of $x^{2}-6x-2=0$, with $\alpha>\beta$. if $a_{n}=\alpha^{n}-\beta{n}$ for $n\geq 1$,then the value of $\frac{a_{10}-2a_{8}}{2a_{9}}$ is \hfill (2011)
\begin{multicols}{4}
\begin{enumerate}
    \item 1
    \item 2
    \item 3
    \item 4    
\end{enumerate}
    
\end{multicols}
\item A value of b for which the equations 

$x^{2}+bx-1=0$

$x6{2}+x+b=0$

have one root in common is \hfill (2011)
\begin{multicols}{2}
\begin{enumerate}
    \item $-\sqrt{2}$
    \item $-i\sqrt{3}$
    \item $i\sqrt{5}$
    \item $\sqrt{2}$
\end{enumerate}
    
\end{multicols}
\item The quadratic equation $p\brak{x}=0$ with real coefficients has purely imaginary roots. Then the equation$p\brak{p\brak{x}}=0$ has \hfill (JEE Adv,2014)
\begin{enumerate}
    \item one purely imaginary root
    \item all real roots
    \item two real roots and two purely imaginary roots 
    \item neither real nor imaginary roots
\end{enumerate}
\item let $-\frac{\pi}{6}<\theta<-\frac{\pi}{12}$. Suppose $\alpha_{1}and\beta_{1}$ are the roots of the equation $x^{2}-2x\sec \alpha+1=0$ and $\alpha_{2}and\beta_{2}$ are the roots of the equation $x^{2}-2x\tan \theta-1=0$. If $\alpha_{1}>\beta_{1}$ and $\alpha_{2}>\beta_{2}$,then $\alpha_{1}+\beta_{2}$ equals \hfill (JEE Adv.2016)
\begin{enumerate}
    \item $2\brak{\sec\theta-\tan\theta}$
    \item $2\sec\theta$
    \item $-2\tan\theta$
    \item 0
    
\end{enumerate}

% \end{enumerate}